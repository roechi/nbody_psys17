\documentclass{beamer}
\usetheme{CambridgeUS}


% Set Color ==============================

% Custom colors
\usepackage{xcolor}

% http://www.computerhope.com/htmcolor.htm
\definecolor{htwgreen}{HTML}{76B900}
\definecolor{deep sky blue}{HTML}{3BB9FF}
\definecolor{light sky blue}{HTML}{82CAFA}

\makeatletter
\definecolor{mybackground}{HTML}{82CAFA}
\definecolor{myforeground}{HTML}{0000A0}

\setbeamercolor{normal text}{fg=black,bg=white}
\setbeamercolor{alerted text}{fg=red}
\setbeamercolor{example text}{fg=black}

\setbeamercolor{background canvas}{fg=myforeground, bg=white}
\setbeamercolor{background}{fg=myforeground, bg=mybackground}

\setbeamercolor{palette primary}{fg=black, bg=gray!30!white}
\setbeamercolor{palette secondary}{fg=black, bg=gray!20!white}
\setbeamercolor{palette tertiary}{fg=black, bg=htwgreen}
\setbeamercolor{frametitle}{fg=black}
\setbeamercolor{title}{fg=black}
\setbeamercolor{section number projected}{bg=gray!20!white,fg=htwgreen}
\setbeamertemplate{section in toc}[square]
\setbeamercolor{subsection number projected}{bg=gray!20!white,fg=htwgreen}
\setbeamertemplate{subsection in toc}[square]

\setbeamertemplate{itemize items}[square]
\setbeamercolor{itemize item}{bg=gray!20!white,fg=htwgreen}
\setbeamercolor{itemize subitem}{bg=gray!20!white,fg=htwgreen}

\makeatother


\usefonttheme{professionalfonts} % using non standard fonts for beamer
\usefonttheme{serif} % default family is serif

\usepackage{fontspec}
\setmainfont[ItalicFont=HTWBerlin Office Italic]{HTWBerlin Office}


\usepackage{graphicx}
\usepackage{amsmath}
\usepackage{ wasysym }
\usepackage{algorithm}% http://ctan.org/pkg/algorithms
\usepackage{algpseudocode}% http://ctan.org/pkg/algorithmicx

\begin{document}
\title{N-Körper Simulation}
\subtitle{VL Parallele Systeme, SoSe2017}
\author{Richard Remus, Jonas Jaszkowic}
\date{28. Juni 2017}

\begin{frame}
\title{N-Körper Simulation}
\titlepage
\end{frame}

\section{Problemstellung}
\begin{frame}
\begin{itemize}
  \item Interaktion von $N$ Körpern im Raum
  \item Gegenseitige Anziehung durch Gravitation
  \item Simulation der Bewegung
\end{itemize}
\end{frame}

\begin{frame}
Grundlage: \textbf{Newton'sche Gesetze der Bewegung}\vspace{1cm}
\begin{itemize}
  \item Trägheitsprinzip
  \item Aktionsprinzip
  \item Wechselwirkungsprinzip
  \item Superpositionsprinzip
\end{itemize}
\end{frame}


\section{Algorithmus}
\begin{frame}
\begin{center}
\begin{huge}
  \begin{align*}
    \vec{a}^{\,}_i = \gamma \sum\limits_{j=1}^N \frac{m_j  \vec{r}^{\,}_{ij}}{(|| \vec{r}^{\,}_{ij} ||^2 + \epsilon^2)^{\frac{3}{2}}}
  \end{align*}  
\end{huge}
\end{center}
\end{frame}

\begin{frame}


\begin{algorithm}[H]
\caption{Update body positions}
\begin{algorithmic}[1]
\Procedure{UpdateStep}{}
   \For{each body $a$} 
      \State $f_a \gets 0$ \Comment{Reset force for $a$}
      \For{each body $b \neq a$}
        \State $f_a \gets f_a + \vec{F}^{\,}_{b\rightarrow a}$ \Comment{Accumulate forces}
      \EndFor
   \EndFor
   \For{each body $a$} 
      \State $v_a \gets v_a + \Delta * f_a$ \Comment{Update velocity}
      \State $r_a \gets r_a + \Delta * v_a$ \Comment{Update position}
   \EndFor
\EndProcedure
\end{algorithmic}
\end{algorithm}


\end{frame}

\section{Implementierung}

\subsection{OMP}
\begin{frame}

\small{\texttt{Q.enqueueNDRangeKernel(K,NullRange,NDRange(1024),NDRange(128));}}
\begin{itemize}
  \item Offset
  \item Global Size
  \item Local Size
  \item The global size (GSZ) is the total number of work-items (WI)
  \item The local size (LSZ) is the number of work-items per work-group (WI/WG)
  \item The number of work-groups is the global size / local size, or GSZ/LSZ, or WG
\end{itemize}
\end{frame}

\subsection{OpenCL}
\begin{frame}
\begin{itemize}
  \item OpenCL Zeugs
\end{itemize}
\end{frame}

\subsection{Softwaredesign}
\begin{frame}
\begin{itemize}
  \item Softwaredesign Zeugs
\end{itemize}
\end{frame}

\subsection{Visualisierung}
\begin{frame}
\begin{itemize}
  \item Visualisierung Zeugs
\end{itemize}
\end{frame}

\section{Benchmarks}
\begin{frame}
\begin{itemize}
  \item Die ganzen Benchmarks
\end{itemize}
\end{frame}

\end{document}